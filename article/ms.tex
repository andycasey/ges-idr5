%% Copyright 2016 Andrew R. Casey.  All rights reserved.

\documentclass[preprint]{aastex}

\usepackage{amsmath}
\usepackage{bm}

\IfFileExists{vc.tex}{\input{vc.tex}}{\newcommand{\githash}{UNKNOWN}\newcommand{\giturl}{UNKNOWN}}

\newcommand{\acronym}[1]{{\small{#1}}}

\newcommand{\ges}{\acronym{GES}}
\newcommand{\project}[1]{\textsl{#1}}

\newcommand{\gaiaeso}{\project{Gaia-ESO}}

\newcommand{\teff}{T_{\mathrm{eff}}}
\newcommand{\logg}{\log g}
\newcommand{\feh}{[\mathrm{Fe/H}]}
\newcommand{\normal}[2]{\mathcal{N}\left(#1, #2\right)}

\begin{document}
\title{The Gaia-ESO Survey: Combined estimates from multiple stellar analysis pipelines}

\author{
    Andrew~R.~Casey\altaffilmark{1},
   }


\begin{abstract}
  Some stuff
\end{abstract}


\keywords{stars: fundamental parameters --- stars: abundances
% Clear rest of page
\clearpage}


\section{Introduction} 
\label{sec:introduction}

% In terms of telescope time cost, GES is the largest by far.

% Huge time allocation to do heaps of cool shit.

% Analyse all spectral types

% Provides a ground-based complement to the Gaia mission.

% Huge collaboration, most of which were in-fighting previously.

% Through GES they started working together.... 

% Experts in different fields, etc.

% Multiple analysis strategy.


\section{Data}
\label{sec:data}

% Description of the setups

% Summary of the ges type and SNR of the data.

% Description of target sources included: MW, Clusters, CoRot, K2, etc.

% Blind test.

\section{Methods}
\label{sec:methods}

% Multiple analysis strategy: why

% primary advantage and disadvantage to multiple analysis strategy

% Description of two parameter rounds: stellar parameters, abundances

% What's reported by each node? Flags? Consistency of flags?

% Description of pipelines in appendix 

% Commonalities between nodes: model atmospheres, line list, normalized spectra, initial params, spectrum library, xi calibration if necessary.



% EW method and comparisons -- into an appendix?.


% What do the nodes provide? Most parameters, flags, etc.

% CONTINUE from here:
% Stellar parameter determination


% Stellar parameter homogenisation
\subsection{Stellar parameter homogenisation}
\label{sec:stellar-parameter-homogenisation}


Flag propagation.

\noindent{}We make the following assumptions when constructing a model for homogenisation:

\begin{itemize}
    \item   \emph{The uncertainties reported by each node are incorrect.}
            This assumption is based on a cursory examination of the data:
            some nodes do not provide uncertainties in stellar parameters, and
            the uncertainties reported by some nodes could vary in percent by
            two orders of magnitude. Therefore, we assume that each node only
            provides a point estimate for astrophysical parameters.

    \item   \emph{Every node provides a biased estimate of stellar parameters.}
            We assume that every node has an unknown bias $\beta$ (offset) in each
            astrophysical parameter, 

            \begin{eqnarray}
                \mu_{true} = \mu_{node} + \beta_{node} + \epsilon
            \end{eqnarray}

            \noindent{}where $\epsilon$ represents the noise in the estimate.

    \item   \emph{The unbiased stellar parameters from any two nodes are correlated.}
            After accounting for individual node biases, we assume that the 
            stellar parameters reported by any two nodes will be correlated
            to some degree. The level of correlation between two nodes is unknown.
            This assumption arises from commonalities that led to the estimates from
            each node. Specifically: all nodes are instructed to use a common list 
            of line transitions, the same MARCS model photospheres, some nodes employ
            similar methods (e.g., spectral synthesis), and all nodes derive their 
            estimates from the same data (in some instances, even the same 
            continuum-normalized spectra are used).

            We note that even two nodes with identical methods are not guaranteed
            to be perfectly correlated, because the \emph{choice} of which transitions
            or spectral regions to use is not prescribed.

    \item   \emph{There are gaps in the data.} 
            No spectrum is guaranteed to have stellar parameters reported by all
            nodes. This assumption is a consequence of the different expertise in
            \gaiaeso: no node or method is expert across the entire range of parameter
            space explored. If stellar parameters are not reported, then the reason 
            must be specified through a \texttt{TECH} flag (see Section \ref{sec:flags}
            on flag usage).

    \item   \emph{Random uncertainties in stellar parameters will increase with
            decreasing S/N ratio.}

    \item   \emph{All `well-studied' stars have noisy measurement, and their values
            are not known with infinite precision.} But we assume they are unbiased!

    \item   \emph{The systematic uncertainty will vary across parameter space.}

    \item   \emph{All stellar parameter estimates are normally-distributed draws about 
            the true astrophysical value.}

    \item   \emph{The effective temperature $\teff$, surface gravity $\logg$, and
            metallicity $\feh$ for a single star are uncorrelated.}
            This assumption is provably incorrect, and is made for practical reasons
            only. In Section \ref{sec:scaling-the-model} we describe how the number
            of model parameters scales, and in Section \ref{sec:model-parameterisation}
            we explain how the inference problem itself is numerically unstable.
            Including correlations in stellar parameters would require a constant
            correlation term $\rho$, or modeling $\rho$ as a function of the stellar
            parameters. In either case, the number of model parameters that require
            MCMC will immediately increase by at least factor of three (e.g.,
            $\gtrsim5{,}000$ parameters), amplifying numerical stability issues and
            substantially increasing the requisite number of MCMC samples. For these
            reasons, simultaneously modeling the correlation between stellar
            parameters has been left for an extension of this work.
 
\end{itemize}

Given these assumptions the model we adopt is
\begin{eqnarray}
    \mu_{bm} = \normal{\mu_{true}}{\sigma_{bm}} 
\end{eqnarray} 


% List of the model parameters

We list the nomenclature and descriptions for all model parameters in Table 
\ref{tab:model-parameters}. The number of total model parameters is given by
\begin{eqnarray}
    N_{parameters} = N_{calibrators} + N_{missing} + 6N_{nodes} + \frac{N_{nodes}!}{2(N_{nodes} - 2)!}
\end{eqnarray}
where $N_{calibrators}$ accounts for the true values of the calibrators, the
$N_{missing}$ models the missing data points for the calibrator spectra, the
$6N_{nodes}$ summarises parameters that model biases, as well as systematic
and random uncertainties. The final term $\frac{N_{nodes}!}{2(N_{nodes} - 2)!}$
is the number of correlation coefficients between all nodes.\footnote{In practice
we include the Cholesky decomposition $L$ of the $N \times N$ correlation matrix $\rho$
as model parameters instead of directly modeling the $\rho$ terms.  The correlation
matrix can be reconstructed by $\rho = (IL)\cdot(IL)^T$ where $I$ is the $N \times N$
identity matrix. This parameterisation helps ensure that the resulting covariance
matrix $\mathbf{\Sigma}$ is positive semi-definite, and can therefore be used for
sampling.} 
 


 

% Priors 

% How does the model scale?

% The implementation details.
% Inheritly unstable
% Cholesky decomposition




% Abundance determination

% Abundance homogenisation


\section{Discussion}
\label{sec:discussion}

% Is the multiple-analysis strategy justified?
% --> Take any one node, infer their uncertainties as a function of SNR using point-wise correlations (e.g., no benchmarks).
% --> What goes wrong for each node?

% Precision, accuracy, systematics.
% EW


% Analysis choices: we give same line list but not all are used.
%                   or how they are used.

 



\section{Conclusions}
\label{sec:conclusions}

% This should be a "lessons learned" list.

% The final paragraph should contain a succint forward outlook of how good or bad an idea this was.


\end{document}
