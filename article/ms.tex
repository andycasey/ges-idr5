%% Copyright 2016 Andrew R. Casey.  All rights reserved.

\documentclass[preprint]{aastex}

\usepackage{amsmath}
\usepackage{bm}

\IfFileExists{vc.tex}{\input{vc.tex}}{\newcommand{\githash}{UNKNOWN}\newcommand{\giturl}{UNKNOWN}}

\newcommand{\acronym}[1]{{\small{#1}}}
\newcommand{\project}[1]{\textsl{#1}}
\newcommand{\article}{\textit{Article}}

\newcommand{\ges}{\acronym{GES}}
\newcommand{\gaiaeso}{\project{Gaia-ESO}}

\newcommand{\teff}{T_{\mathrm{eff}}}
\newcommand{\logg}{\log g}
\newcommand{\feh}{[\mathrm{Fe/H}]}
\newcommand{\normal}[2]{\mathcal{N}\left(#1, #2\right)}

\begin{document}
\title{The \gaiaeso\ Survey: Combined estimates from multiple stellar analysis pipelines}

\author{
    Andrew~R.~Casey\altaffilmark{1},
   }


\begin{abstract}
	% The GES scientific goals
	% GES has a unique analysis strategy.
	% Allows XYZ benefits.
	% Here we describe the process for homogenisation of stellar parameters for FGK-type
	% stars analysed with both UVES and GIRAFFE data, in order to produce recommended results.
	% We construct a noise model for each node, and infer error behaviour for each node as a
	% function of S/N and stellar parameters.
	% We combine these to produce results.
	% For the first time, we describe blind tests conducted in GES to examine subjectivity, etc.
	% The typical reported uncertainties in stellar parameters are XYZ
	% We outline a list of recommended best practices and lessons learned for stellar parameter determination
	% We provide a common, homogeneous metallicity scale for GCs and OCs, and demonstrate that GES results can be used to infer diffusion/settling in clusters??
	% Other potential science highlights:
	% --> Something with velocity/metallicity in the field? two-point correlation function.
	% --> High-/low-alpha sequence with GIRAFFE stuff.
	
\end{abstract}


\keywords{stars: fundamental parameters --- stars: abundances
% Clear rest of page
\clearpage}


\section{Introduction} 
\label{sec:introduction}

% In terms of telescope time cost, GES is the largest by far.

% Huge time allocation to do heaps of cool shit.

% Analyse all spectral types

% Provides a ground-based complement to the Gaia mission.

% Huge collaboration, most of which were in-fighting previously.

% Through GES they started working together.... 

% Experts in different fields, etc.

% Multiple analysis strategy.


\section{Data}
\label{sec:data}

% Description of the setups

% Summary of the ges type and SNR of the data.

% Description of target sources included: MW, Clusters, CoRot, K2, etc.

% Blind test.

\section{Methods}
\label{sec:methods}

The unique parallel analysis strategy adopted by the \ges\ provides a number of
benefits over other spectroscopic surveys.  We can identify the strengths and
weaknesses of state-of-the-art analysis methods, quantify the systematic 
differences that arise from \emph{defensible analysis decisions}, and produce an 
unbiased estimator of stellar parameters with representative uncertainties that account
for both random and systematic contributions.  Although this procedure requires an
enormous amount of collaboration and coordination, it is necessary to provide
a thorough evaluation of systematic uncertainties present between analysis 
pipelines that are routinely used in the literature.


We perform the \ges\ spectroscopic analysis for each data release in two 
consecutive rounds.  In the first round, the stellar parameters are reported by
each analysis group.  The homogenisation of stellar parameters is performed by the
working group leads (WG10, WG11, WG12, WG13), and the recommended results from each
working group are submitted to the top-level homogenisation group, WG15. WG15 makes
checks to ensure the results from different working groups are on the same scale, 
and combines the results from multiple working groups to produce a single set of 
\emph{Survey}-level recommended stellar parameters for each star \citep{Hourihane:2017}.
These recommended results are used in the second analysis round, where individual 
nodes report detailed chemical abundances given the recommended stellar parameters.
In this \article\ we describe only the first round of analysis; the homogenisation 
of detailed chemical abundances for FGK-type stars will be described in a companion 
contribution (Casey et al., 2017).


\subsection{Analysis methods}
\label{sec:analysis-nodes}
% How many nodes contributed to DR5 in WG11/WG10, and how many results did they provide per setup? (what fraction?)

% Summary description of nodes
% What do the nodes provide? Most parameters, flags, etc.
% Detailed description of pipelines in appendix 

% EW method and comparisons -- into an appendix?.

\subsubsection{Summary of (dis-)similarities in analysis methods}
\label{sec:analysis-node-similarities}

Despite the multiple analysis strategies adopted in \ges, there are a number of
commonalities between the nodes. Specifically, all FGK-analysis nodes use the same:
\begin{enumerate}
  \item Updated MARCS model photospheres \citep{BG}.
  \item Atomic and molecular line lists, updated with recently measured oscillator strengths \citep{Heiter:2016,someone}.
  \item Rest-frame, pseudo-continuum-normalised spectra are provided to all nodes by the data reduction group. Individual echelle orders are also provided. Inverse variance arrays are available for all flux arrays.
  \item Initial stellar parameters are provided for all stars observed with GIRAFFE.
  \item Infrared photometry is available for all targets.
  \item For methods that perform template-fitting, all nodes are provided with a grid of high-resolution continuum-normalised spectra synthesised using (1) and (2).  The synthetic grid covers the entire wavelength region used for \ges.
  \item Where relevant, nodes are provided with empirical microturbulence calibrations for main-sequence, sub-giant, and giant stars. These relations are given in Appendix \ref{app:xi}.
\end{enumerate}


There are three caveats to the listed items above.  Firstly, while nodes are encouraged
to use the same empirically-scaled interpolator for photospheric models \citep{Masseron:2006},
this is not required.  Secondly, the nodes are not required to use the 
pseudo-continuum-normalised spectra provided to them; each node can chose to 
adopt their own continuum normalisation procedure.  Lastly, although all nodes use the 
same atomic and molecular data, the nodes are free to chose \emph{which} lines (or 
spectral regions) to use when fitting their models to the data. 


In summary, the analysis nodes generally differ only in \emph{defensible analysis decisions}:
there is no well-founded \emph{a priori} reason that would explain why one method should be
significantly better or worse across the space of stellar parameters.  For these reasons,
the Gaia-ESO Survey is well-placed to critically examine what analysis decisions will
objectively result in more accurate or precise results.


\subsection{Flag propagation}
\label{sec:flag-propagation}

In earlier data releases the analysis nodes were \emph{encouraged} to submit flags that
describe technicalities that may have affected their analysis, or peculiar issues
that were identified in a star or spectrum.  These flags proved invaluable in enabling 
homogenisation leads to identify scenarios where particular pipelines perform poorly. 
For example, OACT (WG11) were able to identify and account for high rotation, whereas 
ULB (WG11) would not identify fast rotators, and ULB (WG11) would report significantly
noisier stellar parameters for fast rotators.


From DR5 onwards, these examples led to the decision to \emph{require} analysis nodes 
to submit stellar parameters ($\teff$, $\logg$, [Fe/H], and $\xi$ where appropriate),
for every spectrum, or supply a technical flag explaining why no stellar parameters
could be derived (preferably supply both when appropriate). The full list of flags used
in DR5 is listed in Appendix \ref{appendix:flags}.  In Figures \ref{fig:flag-propagation-wg10}
and \ref{fig:flag-propagation-wg11} we show heatmaps of flag usage between different 
analysis nodes in WG10 and WG11, respectively.  If all nodes reported the same flags then
a one-to-one density mapping would be visible.  Note that here we are only showing instances
where multiple flags have been reported for a single spectrum: for the sake of clarity we
do not show situations where only one flag is reported by a single node.  Figures 
\ref{fig:flag-propagation-wg10} and \ref{fig:flag-propagation-wg11} demonstrate that when one analysis node reports a flag, it
is highly unlikely that the same issue has been reported by another node. The most common
situation is that one node will report a descriptive flag (e.g., \texttt{10106} 
to indicate a broken spectrum containing a picket-fence or Heaviside pattern), and other
nodes will report a non-specific flag (e.g., \texttt{10302}: code convergence issue).


We constructed a sequence of rules to propagate detailed flag information to other results.
We categorised these rules to propagate to results that were derived from either the same 
spectrum (e.g., \texttt{10106} broken spectrum), or the same star (e.g., suspected binary 
star system).  These propagation rules are listed in Tables \ref{tab:flag-propagation-spectrum}
and \ref{tab:flag-propagation-cname}. Any results with propagated technical flags were 
marked as failing internally quality control standards, and were not used in the homogenisation
process.  Results with any of the flags listed in Table \ref{tab:node-specific-flags} were
were also excluded from homogenisation, however those flags were considered `node-specific'
in that they do not necessarily require propagation to other results derived from the same
spectrum or star.


We also excluded all cluster records (\texttt{GES\_TYPE} of 'XX-XX-XX' or `XX-XX-XX') 
reported by OACT (WG11).  Although OACT performed excellently in the benchmark and blind tests
in WG11, we identified convergence issues in cluster field, which consequently altered the
metallicity scale of globular clusters.

% Any other exclusions? WG10 IAC results at edges?

The total number of records (set of $\teff$, $\logg$, [Fe/H]) that were marked as failing
internal quality controls due to reported or propagated flags were XXX for WG10 and XXX for
WG11.  This represents a small fraction of the total number of records submitted in each
working group (XXX\% for WG10 and XXX\% for WG11).  The number of stars (unique \texttt{CNAME}s) where all results were excluded due to reported or propagated flags was XXX for WG10 and XXX for WG11.  These cases were principally due to data reduction issues (e.g., flags 
\texttt{10100-10150}), and these examples were supplied to the data reduction group to
examine before the sixth data release.



\subsection{Stellar parameter homogenisation}
\label{sec:stellar-parameter-homogenisation}


\noindent{}We make the following assumptions in our homogenisation model:

\begin{itemize}
    \item   \emph{The uncertainties reported by each node are incorrect.}
            This assumption is based on a cursory examination of the data:
            some nodes do not provide uncertainties in stellar parameters, and
            the uncertainties reported by some nodes could vary in percent by
            two orders of magnitude. Therefore, we assume that each node only
            provides a point estimate for astrophysical parameters.

    \item   \emph{Every node provides a biased estimate of stellar parameters.}
            We assume that every node has an unknown bias $\beta$ (offset) in each
            astrophysical parameter, 

            \begin{eqnarray}
                \mu_{true} = \mu_{node} + \beta_{node} + \epsilon
            \end{eqnarray}

            \noindent{}where $\epsilon$ represents the noise in the estimate.

    \item   \emph{The unbiased stellar parameters from any two nodes are correlated.}
            After accounting for individual node biases, we assume that the 
            stellar parameters reported by any two nodes will be correlated
            to some degree. The level of correlation between two nodes is unknown.
            This assumption arises from commonalities that led to the estimates from
            each node (Section \ref{sec:analysis-node-similarities}).
            
            We reiterate that even two nodes with identical methods are not guaranteed
            to be perfectly correlated, because the \emph{choice} of which transitions
            or spectral regions to use is not prescribed.

    \item   \emph{There are gaps in the data.} 
            No spectrum is guaranteed to have stellar parameters reported by all
            nodes. This assumption is a consequence of the different expertise in
            \gaiaeso: no node or method is expert across the entire range of parameter
            space explored. 
            
    \item   \emph{Random uncertainties in stellar parameters will increase with
            decreasing S/N ratio.} 

    \item   \emph{All `well-studied' stars have noisy measurement, and their values
            are not known with infinite precision.} That is to say, we assume that there
            are some well-studied stars with unbiased measurements, but those
            measurements are noisy. For this reason, we must include the stellar
            parameters of each well-studied star as a model parameter, and use 
            non-spectroscopic data to place a strong prior on that stellar parameter.
            
    \item   \emph{The systematic uncertainty will vary across parameter space.}

    \item   \emph{All stellar parameter estimates are normally-distributed draws about 
            the true astrophysical value.}

    \item   \emph{The effective temperature $\teff$, surface gravity $\logg$, and
            metallicity $\feh$ for a single star are uncorrelated.}
            This assumption is provably incorrect, and is made for practical reasons
            only. In Section \ref{sec:scaling-the-model} we describe how the number
            of model parameters scales, and in Section \ref{sec:model-parameterisation}
            we explain how the inference problem itself is numerically unstable.
            Including correlations in stellar parameters would require a constant
            correlation term $\rho$, or modeling $\rho$ as a function of the stellar
            parameters. In either case, the number of model parameters that require
            MCMC will immediately increase by at least factor of three (e.g.,
            $\gtrsim5{,}000$ parameters), amplifying numerical stability issues and
            substantially increasing the requisite number of MCMC samples. For these
            reasons, simultaneously modeling the correlation between stellar
            parameters has been left for an extension of this work.
 
\end{itemize}

Given these assumptions the model we adopt is
\begin{eqnarray}
    \mu_{bm} = \normal{\mu_{true}}{\sigma_{bm}} 
\end{eqnarray} 


% List of the model parameters

We list the nomenclature and descriptions for all model parameters in Table 
\ref{tab:model-parameters}. The number of total model parameters is given by
\begin{eqnarray}
    N_{parameters} = N_{calibrators} + N_{missing} + 6N_{nodes} + \frac{N_{nodes}!}{2(N_{nodes} - 2)!}
\end{eqnarray}
where $N_{calibrators}$ accounts for the true values of the calibrators, the
$N_{missing}$ models the missing data points for the calibrator spectra, the
$6N_{nodes}$ summarises parameters that model biases, as well as systematic
and random uncertainties. The final term $\frac{N_{nodes}!}{2(N_{nodes} - 2)!}$
is the number of correlation coefficients between all nodes.\footnote{
This inference problem is numerically unstable because a \emph{positive semi-definite}
covariance matrix must be constructed from the model parameters in every MCMC draw, 
and there is no joint prior available to ensure that the positive semi-definite 
condition is met.  Although there are parameterisations that will help enforce 
this condition \citep{Pinheiro:1996}, in practice we found that including 
the Cholesky decomposition $L$ of the $N \times N$ correlation matrix $K$ as model
parameters --- instead of directly modelling the set of node correlation 
coefficients $\{\rho\}$ --- was sufficient to ensure positive-semidefinite 
covariance matrices and produce sensible solutions.  The correlation matrix can 
then be reconstructed by $K = (IL)\cdot(IL)^T$ where $I$ is the $N \times N$ identity
matrix.  This representation implies a uniform prior on all correlation coefficients
$p\left(\rho\right) = \mathcal{U}\left(-1, +1\right)$.}

% Priors



We implemented the model in \texttt{Stan}, a probabilistic programming
language, and the code describing our parameterisation is included as Appendix \ref{
appendix:stan-code}. We initialised the model with $\mu_{true}$ at the central
calibrator values, and set XXXXXX, with no correlation between any nodes. We
optimized all parameters in \texttt{Stan} using the Broyden-Fletcher-Goldfarb-Shanno
algorithm. The optimized point was used as an initialisation point for Markov Chain
Monte-Carlo (MCMC) sampling.

N chains, thinning, N iterations. Convergence criteria.
 
 




% Abundance determination

% Abundance homogenisation


\section{Discussion}
\label{sec:discussion}

% Is the multiple-analysis strategy justified?
% --> Take any one node, infer their uncertainties as a function of SNR using point-wise correlations (e.g., no benchmarks).
% --> What goes wrong for each node?

% Precision, accuracy, systematics.
% EW


% Analysis choices: we give same line list but not all are used.
%                   or how they are used.

 



\section{Conclusions}
\label{sec:conclusions}

% This should be a "lessons learned" list.

% The final paragraph should contain a succint forward outlook of how good or bad an idea this was.

\begin{thebibliography}{}
\bibitem[Pinheiro \& Bates(1996)]{Pinheiro:1996} Pinheiro, J.~C., \& Bates, D.~M.\ 1996, Statistics and Computing, 6, 3


\end{thebibliography}

\newpage
\appendix
\section{Microturbulence calibrations} \label{app:xi}

% Version 2.0 taken from http://great.ast.cam.ac.uk/GESwiki/GesWg/GesWg11/Microturbulence

The following microturbulence calibrations were recommended for nodes to use where appropriate. For main-sequence or sub-giant stars with $\teff >= 5000$~K and $\logg >= 3.5$,

\begin{equation}
\xi = 1.05 + 2.51\times10^{-4}(\teff T_0) + 1.5\times10^{-7}(\teff - T_0)^2 - 0.14(\logg - g_0) - 0.005(\logg - g_0)^2 + 0.05\feh + 0.01\feh^2 \quad ,
\end{equation}

\noindent{}and for main-sequence or sub-giant stars with $\teff < 5000$~K,

\begin{equation}
\xi = 1.05 + 2.51\times10^{-4}(5000 - T_0) + 1.5\times10^{-7}(5000 - T_0)^2 - 0.14(\logg - g_0) - 0.005(\logg - g_0)^2 + 0.05\feh + 0.01\feh^2 \quad ,
\end{equation}

\noindent{}and for giant stars:

\begin{equation}
\xi = 1.25 + 4.01\times10^{-4}(\teff - T_0) + 3.1\times10^{-7}(\teff - T_0)^2 - 0.14(\logg - g_0) - 0.05(\logg - g_0)^2 + 0.05\feh + 0.01\feh^2 \quad .
\end{equation}

\noindent{}In all three cases above: $T_0 = 5500$~K and $g_0 = 4.0$.


\section{Flags} \label{app:flags}

\section{\texttt{Stan} code for homogenisation noise model} \label{app:stan-code}

\end{document}
